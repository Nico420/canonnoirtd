% !TEX TS-program = pdflatex
% !TEX encoding = UTF-8 Unicode

% This is a simple template for a LaTeX document using the "article" class.
% See "book", "report", "letter" for other types of document.

\documentclass[11pt]{article} % use larger type; default would be 10pt

\usepackage[utf8]{inputenc} % set input encoding (not needed with XeLaTeX)
\usepackage[T1]{fontenc}
\usepackage{times}

\usepackage{graphicx} % support the \includegraphics command and options

\usepackage[francais]{babel}

\usepackage{listings}

\usepackage{url}

\usepackage{rotating}

\usepackage[a4paper]{geometry}


\date{\today}

\title{Rapport Travaux Pratiques : \\Programmation par Contraintes\\ - TP 5 : \\\textbf{Contraindre puis chercher}}
\author{Nicolas Desfeux\\Aurélien Texier}
\begin{document}
\lstset{language=Prolog,breaklines=true,numbers=left,basicstyle=\footnotesize ,numberstyle=\footnotesize}
\maketitle
\paragraph{} Dans ce TP, nous allons chercher à résoudre un problème de gestion de production. Pour cela, nous allons utiliser la programmation par contraintes.

\section{Le problème}
Le problème que nous allons résoudre concerne une unité de production pour différentes gammes de téléphones mobiles.\\
L'objectif est de trouver quels fabrications il faut lancer pour obtenir un bénéfice maximum.\\
Voici les données qui nous sont fournies : \\
Pour chaque type gamme, nous avons : 
\begin{itemize}
\item Le nombre de techniciens de la gamme,
\item La quantité de téléphone que l'on peut produire par jour,
\item Le bénéfice obtenu.
\end{itemize}

\section{Modéliser et contraindre}
\paragraph{Question 5.1}
On définit ici différents prédicats qui créent l'ensembles des données et des variables du problème.
\lstinputlisting[caption="Définition des 3 vecteurs de valeurs et du vecteur de variables"]{defVect.pl}
\paragraph{Question 5.2}
Le code suivant permet de définir les prédicats permettant d'exprimer : 
\begin{itemize}
\item Le nombre toal d'ouvriers nécéssaire,
\item Le vecteur de bénéfice total par sorte de téléphones,
\item Le profit total.
\end{itemize}
\lstinputlisting[caption="Différents prédicats utiles à la résolution du problème"]{opVect.pl}
\paragraph{Question 5.3} 
On a défini le prédicat pose\_contraintes comme une succesion de prédicats. Contrairement aux T.P. précédents, nous n'avons pas créé de prédicat solve, qui aurait appelé (entre autre), pose\_contrainte.
\lstinputlisting[caption="Prédicat \textbf{pose\_contraintes}"]{poseContrainte.pl}
\section{Optimiser}
\subsection{Branch and bound dans ECLiPSe}
\paragraph{Question 5.4} Il faut toujours faire un labeling dans le but,
car il faut que les variables aient ete instanciees.
\lstinputlisting[caption="Prédicat \textbf{test} avec minimize"]{5-4.pl}


\subsection{Application à notre problème}
\paragraph{Question 5.5} Pour trouver le profit maximum, on crée une variable que l'on cherche à minimaliser, et l'on pose comme contrainte qu'elle soit égale à l'opposé du profit. On obtient ainsi le profit maximum.
\lstinputlisting[caption="Prédicat \textbf{pose\_contrainte} avec maximisation du profit"]{5-5.pl}
Nous avons choisi d'appeler le minimize lors de l'appel de la fonction principale.
\paragraph{Question 5.6}
Pour cette question, il nous est demandé de changer la politique de l'entreprise. On choisit de créer un seuil pour le profit, et minimiser le nombre d'employés.
\lstinputlisting[caption="Prédicat \textbf{pose\_contrainte2} avec seuil pour le profit et minimisation du nombre d'ouvriers"]{5-6.pl}
Pour obtenir ce résultat, nous avons juste rajouter une contrainte pour le seuil du profit, et une recherche du minimum pour le nombre d'ouvrier.
\newpage

\section{Code Complet, avec l'ensemble des tests}
\lstinputlisting[caption="TP5"]{tp5.pl}
\end{document}
