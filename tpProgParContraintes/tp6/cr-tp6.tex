% !TEX TS-program = pdflatex
% !TEX encoding = UTF-8 Unicode

% This is a simple template for a LaTeX document using the "article" class.
% See "book", "report", "letter" for other types of document.

\documentclass[11pt]{article} % use larger type; default would be 10pt

\usepackage[utf8]{inputenc} % set input encoding (not needed with XeLaTeX)
\usepackage[T1]{fontenc}
\usepackage{times}

\usepackage{graphicx} % support the \includegraphics command and options

\usepackage[francais]{babel}

\usepackage{listings}

\usepackage{url}

\usepackage{rotating}

\usepackage[a4paper]{geometry}


\date{\today}

\title{Rapport Travaux Pratiques : \\Programmation par Contraintes\\ - TP 6 : \\\textbf{Sur une balançoire}}
\author{Nicolas Desfeux\\Aurélien Texier}
\begin{document}
\lstset{language=Prolog,breaklines=true,numbers=left,basicstyle=\footnotesize ,numberstyle=\footnotesize}
\maketitle
\paragraph{} Dans ce TP, nous allons chercher à appliquer ce que nous avons appris à un problème de la vie réelle (aussi vraisemblable soit il !).

\section{Trouver une solution au problème}
\paragraph{Question 6.1} Trouver une solution
\paragraph{Question 6.2} Demander à Eclipse
\paragraph{Question 6.3} Il existe une symétrie entre les solutions du problème. Une solution qui place 5 personnes à droite possède une symétrie si l'on met ces 5 personnes à gauche. Il n'est donc pas forcément utile de les faire rechercher par Eclipse !\\
Pour éliminer cette symétrie, nous avons choisi de créer une nouvelle contrainte

\lstinputlisting{test.pl}
\section{Trouver la meilleure solution}
\paragraph{Question 6.4}
\newpage

\section{Code Complet, avec l'ensemble des tests}
\lstinputlisting[caption="TP6"]{tp6.pl}
\end{document}
