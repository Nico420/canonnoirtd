% !TEX TS-program = pdflatex
% !TEX encoding = UTF-8 Unicode

% This is a simple template for a LaTeX document using the "article" class.
% See "book", "report", "letter" for other types of document.

\documentclass[11pt]{article} % use larger type; default would be 10pt

\usepackage[utf8]{inputenc} % set input encoding (not needed with XeLaTeX)
\usepackage[T1]{fontenc}
\usepackage{times}

\usepackage{graphicx} % support the \includegraphics command and options

\usepackage[francais]{babel}

\usepackage{listings}

\usepackage{url}

\usepackage{rotating}

\usepackage[a4paper]{geometry}


\date{\today}

\title{Rapport Travaux Pratiques : \\Programmation par Contraintes\\ - TP 3 : \\\textbf{Contraintes Logiques}}
\author{Nicolas Desfeux\\Aurélien Texier}
\begin{document}
\lstset{language=Prolog,breaklines=true,numbers=left,basicstyle=\footnotesize ,numberstyle=\footnotesize}
\maketitle
\paragraph{} Dans ce T.P., nous allons utiliser la programmation par contraintes pour résoudre un problème d'ordonnancement de tâches à effectuer sur deux machines.\\Dans un premier temps, nous définirons les prédicats qui fixent les domaines dans lesquels nous travaillerons, puis nous ajouterons les contraintes liées au fait que les tâches doivent être effectuées seulement après que certaines autres soient faites. Enfin, nous finirons par une dernière contraintes qui vise à empêcher que deux tâches se fassent simultanément sur la même machine.

\paragraph{Question 3.1}
Nous définissons ici un prédicat \textit{taches(?Taches)} qui unifie Taches au tableau des tâches.
\lstinputlisting[caption="taches"]{taches.pl}

\paragraph{Question 3.2}
Nous définissons ici un prédicat \textit{affiche(+Taches)} qui affiche chaque élément, à savoir chaque tâche constituant le problème. Nous définiront ce prédicat à l'aide d'un itérateur.
\lstinputlisting[caption="affiche"]{affiche.pl}

\paragraph{Question 3.3}
Nous définissons ici un prédicat \textit{domaines(+Taches,?Fin)} qui contraint chaque tâche à commencer après l'instant 0 et à finir avant Fin, variable qui correspond à l'instant où toutes les tâches sont terminées.
\lstinputlisting[caption="domaines"]{domaines.pl}

\paragraph{Question 3.4}
Voici le prédicat \textit{getVarList(+Taches,?Fin,?List)} qui permet de récupérer la liste des variables du problème.
\lstinputlisting[caption="getVarList"]{getVarList.pl}

\paragraph{Question 3.5}
On définit le prédicat \textit{solve(?Taches,?Fin)} qui permet, en utilisant les trois prédicats précédents, de trouver un ordonnancement qui respecte les contraintes de domaines définies.
\lstinputlisting[caption="solve1"]{solve1.pl}

\paragraph{Question 3.6}
On définit ici un prédicat \textit{precedences(+Taches)} qui contraint chaque tâche à démarrer après la fin de ses tâches préliminaires.\\ On modifie alors solve pour prendre en compte ces contraintes.
\lstinputlisting[caption="precedences"]{precedences.pl}

\paragraph{Question 3.7}
Enfin, on définit le prédicat \textit{conflits(+Taches)} qui impose que, sur une machine, deux tâches ne se déroulent pas en même temps.\\ On modifie solve de la même manière qu'à la question précédente pour obtenir une solution du problème prenant en compte cette dernière contrainte.
\lstinputlisting[caption="conflits"]{conflits.pl}

\paragraph{Question 3.8}
Oui, la solution est la meilleure !
Prolog résoud les contraintes en incrémentant le début des tâches, jusqu'à obtenir le respect des contraintes.
Comme il incrémente, la première trouvée est forcément la solution optimale au problème, à savoir l'ordonnancement des tâches au plus tôt.
\newpage

\section{Code Complet, avec l'ensemble des tests}
\lstinputlisting[caption="TP3"]{tp3.pl}
\end{document}
