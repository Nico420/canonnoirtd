% !TEX TS-program = pdflatex
% !TEX encoding = UTF-8 Unicode

% This is a simple template for a LaTeX document using the "article" class.
% See "book", "report", "letter" for other types of document.

\documentclass[11pt]{article} % use larger type; default would be 10pt

\usepackage[utf8]{inputenc} % set input encoding (not needed with XeLaTeX)
\usepackage[T1]{fontenc}
\usepackage{times}

\usepackage{graphicx} % support the \includegraphics command and options

\usepackage[francais]{babel}

\usepackage{listings}

\usepackage{url}

\usepackage{rotating}

\usepackage[a4paper]{geometry}


\date{\today}

\title{Rapport Travaux Pratiques : \\Programmation par Contraintes\\ - TP 2 : \\\textbf{Contraintes Logiques}}
\author{Nicolas Desfeux\\Aurélien Texier}
\begin{document}
\lstset{language=Prolog,breaklines=true,numbers=left,basicstyle=\footnotesize ,numberstyle=\footnotesize}
\maketitle
\paragraph{} Dans ce T.P., nous allons utiliser les contraintes et les domaines finis pour résoudre un problème. Nous allons tenter, à partir des informations (contraintes) qui nous sont fournies, de déduire d'autres informations.\\
Pour cela, nous allons définir plusieurs prédicats qui nous permettront au final d'obtenir les informations que nous cherchons.

\paragraph{Question 1.1}
Nous allons définir les différents domaines dont nous allons avoir besoin au cours de notre problème.
\lstinputlisting[caption="Domaines"]{domaines}

\paragraph{Question 1.2}
Nous avons également besoin de définir des prédicats qui permettent de contraindre le domaine des variables qui composent une maison.
\lstinputlisting[caption="Domaines maisons"]{domaine_maison}

\paragraph{Question 1.3}
Ce prédicat permet de définir la liste des maisons, tout en posant les différentes contraintes que l'on a précédement définies. C'est également grâce à ce prédicat que l'on pose les numéros des maisons.
\lstinputlisting[caption="Prédicat \textit{\textbf{rue(?Rue)}}"]{rue}

\paragraph{Question 1.4}
Ce prédicat va permettre un affichage clair des maisons.
\lstinputlisting[caption="Prédicat \textit{\textbf{ecrit\_maison(?Rue)}}"]{ecrit_maison}

\paragraph{Question 1.5}
On définit ici un prédicat qui permet de récupérer la liste des variables du problème.\\
On utilise ensuite un prédicat de labeling, qui va permettre de labeliser les variables par rapport aux différents domaines que l'on a définit.
\lstinputlisting[caption="Prédicat \textit{\textbf{getVarList(+List)}}"]{getVarList}

\paragraph{Question 1.6}
Le prédicat resoudre va permettre de trouver une solution respectant les contraintes de domaines, mais pas les contraintes du problèmes.
\lstinputlisting[caption="Prédicat \textit{\textbf{resoudre(?Rue)}}"]{resoudre}
\paragraph{Question 1.7}
Maintenant, nous définissons les différentes contraintes, et un prédicat pouvant les utiliser.
\lstinputlisting[caption="Définition des contraintes et prédicat de résolution"]{final}

\paragraph{Question 1.8}
On obtient donc la réponse aux questions posées par le problème : \\\\
\centering{
\textbf{Le \textit{japonais} possède un \textit{zèbre} et le \textit{norvégien} boit  de \textit{l'eau}}
}
\newpage

\section{Code Complet, avec l'ensemble des tests}
\lstinputlisting[caption="TP2"]{tp2.pl}
\end{document}
